\documentclass{article}

\usepackage{fancyhdr}
\usepackage{extramarks}
\usepackage{amsmath}
\usepackage{amsthm}
\usepackage{amsfonts}
\usepackage{tikz}
\usepackage[plain]{algorithm}
\usepackage{algpseudocode}
\usepackage{enumerate}
\usepackage{tikz}
\usepackage{pythonhighlight}
\usetikzlibrary{automata,positioning}

%
% Basic Document Settings
%  

\topmargin=-0.45in
\evensidemargin=0in
\oddsidemargin=0in
\textwidth=6.5in
\textheight=9.0in
\headsep=0.25in

\linespread{1.1}

\pagestyle{fancy}
\lhead{\hmwkAuthorName}
\chead{\hmwkClass : \hmwkTitle}
\rhead{\firstxmark}
\lfoot{\lastxmark}
\cfoot{\thepage}

\renewcommand\headrulewidth{0.4pt}
\renewcommand\footrulewidth{0.4pt}

\setlength\parindent{0pt}

%
% Create Problem Sections
%

\newcommand{\enterProblemHeader}[1]{
    \nobreak\extramarks{}{Problem \arabic{#1} continued on next page\ldots}\nobreak{}
    \nobreak\extramarks{Problem \arabic{#1} (continued)}{Problem \arabic{#1} continued on next page\ldots}\nobreak{}
}

\newcommand{\exitProblemHeader}[1]{
    \nobreak\extramarks{Problem \arabic{#1} (continued)}{Problem \arabic{#1} continued on next page\ldots}\nobreak{}
    \stepcounter{#1}
    \nobreak\extramarks{Problem \arabic{#1}}{}\nobreak{}
}

\newcommand*\circled[1]{\tikz[baseline=(char.base)]{
		\node[shape=circle,draw,inner sep=2pt] (char) {#1};}}


\setcounter{secnumdepth}{0}
\newcounter{partCounter}
\newcounter{homeworkProblemCounter}
\setcounter{homeworkProblemCounter}{1}
\nobreak\extramarks{Problem \arabic{homeworkProblemCounter}}{}\nobreak{}

%
% Homework Problem Environment
%
% This environment takes an optional argument. When given, it will adjust the
% problem counter. This is useful for when the problems given for your
% assignment aren't sequential. See the last 3 problems of this template for an
% example.
%

\newenvironment{homeworkProblem}[1][-1]{
    \ifnum#1>0
        \setcounter{homeworkProblemCounter}{#1}
    \fi
    \section{Problem \arabic{homeworkProblemCounter}}
    \setcounter{partCounter}{1}
    \enterProblemHeader{homeworkProblemCounter}
}{
    \exitProblemHeader{homeworkProblemCounter}
}

%
% Homework Details
%   - Title
%   - Class
%   - Due date
%   - Name
%   - Student ID

\newcommand{\hmwkTitle}{Homework\ \#07}
\newcommand{\hmwkClass}{Probability \& Statistics for EECS}
\newcommand{\hmwkDueDate}{Apr 28, 2024}
\newcommand{\hmwkAuthorName}{Fei Pang}
\newcommand{\hmwkAuthorID}{2022533153}


%
% Title Page
%

\title{
    \vspace{2in}
    \textmd{\textbf{\hmwkClass:\\  \hmwkTitle}}\\
    \normalsize\vspace{0.1in}\small{Due\ on\ \hmwkDueDate\ at 23:59}\\
	\vspace{4in}
}

\author{
	Name: \textbf{\hmwkAuthorName} \\
	Student ID: \hmwkAuthorID}
\date{}

\renewcommand{\part}[1]{\textbf{\large Part \Alph{partCounter}}\stepcounter{partCounter}\\}

%
% Various Helper Commands
%

% Useful for algorithms
\newcommand{\alg}[1]{\textsc{\bfseries \footnotesize #1}}
% For derivatives
\newcommand{\deriv}[1]{\frac{\mathrm{d}}{\mathrm{d}x} (#1)}
% For partial derivatives
\newcommand{\pderiv}[2]{\frac{\partial}{\partial #1} (#2)}
% Integral dx
\newcommand{\dx}{\mathrm{d}x}
% Alias for the Solution section header
\newcommand{\solution}{\textbf{\large Solution}}
% Probability commands: Expectation, Variance, Covariance, Bias
\newcommand{\E}{\mathrm{E}}
\newcommand{\Var}{\mathrm{Var}}
\newcommand{\Cov}{\mathrm{Cov}}
\newcommand{\Bias}{\mathrm{Bias}}

\begin{document}

\maketitle

\pagebreak

\begin{homeworkProblem}[1]

    \begin{align*}
        (1)P(Y = y|X = x) &= \frac{P(Y = y, X = x)}{P(X = x)} = \frac{P(X = x|Y = y)P(Y = y)}{P(X = x)} \\[10pt]
        (2)f_{Y}(y|X = x) &= \frac{P(y \in (y - \epsilon, y + \epsilon)|X = x)}{2\epsilon} \\
        &= \frac{P(X = x|y \in (y - \epsilon, y + \epsilon))P(y \in (y - \epsilon, y + \epsilon)|X = x)}{2\epsilon P(x = x)} \\
        &= \frac{P(X = x|Y = y)f_{Y}(y)}{P(X = x)} \\[10pt]
        (3)P(Y = y|X = x) &= \frac{P(Y = y, X \in (x - \epsilon, x + \epsilon))}{P(X = x)} \\
        &= \frac{P(X \in (x - \epsilon, x + \epsilon)|Y = y)P(Y = y)}{P(X = x)} \\
        &= \frac{P(X \in (x - \epsilon, x + \epsilon))}{2\epsilon} \frac{P(Y = y)}{P(X \in (x - \epsilon, x + \epsilon))} \\
        &= \frac{f_{X}(X|Y = y)P(Y = y)}{f_{X}(x)} \\[10pt]
        (4)f_{Y|X}(y|x) &= \frac{f_{X,Y}(x, y)f_{Y}(y)}{f_{X}(x)} = \frac{f_{X|Y}(x|y)f_{Y}(y)}{f_{X}(x)}
        \end{align*}
  
\end{homeworkProblem}

\begin{homeworkProblem}[2]

    \begin{enumerate}[(a)]
    \item 
    Since \(N = X + Y\), they are dependent.

    Conditioning on \(N = n\), we know that \(X|_{N=n} \sim \text{Bin}(n,p)\), \(Y|_{N=n} \sim \text{Bin}(n,1-p)\). 

    For \(i,j \geq 0\),
    \begin{align*}
    P(N = n, X = i, Y = j) &= P(X = i, Y = j) \\
    &= \sum_{n=0}^{\infty} P(X = i, Y = j | N = n)P(N = n) \\
    &= P(X = i, Y = j | N = i + j)P(N = i + j) \\
    &= P(X = i | N = i + j)P(Y = j | X = i, N = i + j)P(N = i + j) \\
    &= P(X = i | N = i + j)P(N = i + j) \\
    &= \binom{i+j}{i} p^i (1 - p)^j \cdot \frac{e^{-\lambda}\lambda^{i+j}}{(i+j)!} \\
    &= \frac{e^{-\lambda p}(\lambda p)^i}{i!} \cdot \frac{e^{-\lambda(1-p)}(\lambda(1 - p))^j}{j!} \\
    &= e^{-\lambda} \frac{(\lambda p)^i}{i!} \cdot e^{-\lambda(1-p)} \frac{(\lambda(1 - p))^j}{j!}
    \end{align*}

    \item 
    Since \(N = X + Y\), they are dependent.

    According to (a), \(P(N = n, X = i)\) is \(P(X = i, Y = n - i)\) which simplifies to the given expression.

    \item 
    Since \(N\) is indeterminate, \(X, Y\) are independent.

    According to (a), \(P(X = i, Y = j)\) is given by the product of their individual probabilities, leading to \(X\) being distributed as \(\text{Pois}(\lambda p)\) and similarly \(Y\) as \(\text{Pois}(\lambda(1 - p))\).

    \item 
    For the covariance and correlation calculations, the following equations are used:

    \begin{align*}
    \text{Cov}(N, X) &= \text{Cov}(X + Y, X) \\
    &= \text{Cov}(X, X) + \text{Cov}(Y, X) \\
    &= \text{Var}(X) + \text{Cov}(Y, X) \\
    &= \lambda p \\
    \text{Corr}(N, X) &= \frac{\text{Cov}(N, X)}{\sqrt{\text{Var}(N)\text{Var}(X)}} \\
    &= \sqrt{p}
    \end{align*}

    \end{enumerate}

    \end{homeworkProblem}


\begin{homeworkProblem}[3]

        
\begin{enumerate}[(a)]
    \item
    Since \( X \sim \text{Expo}(\lambda) \), \( Y \sim \text{Expo}(\lambda) \), \( f_X(x) = \lambda e^{-\lambda x}, x \geq 0 \); \( f_Y(y) = \lambda e^{-\lambda y}, y > 0 \).
    Suppose \( T \leq t \),

    When \( t \leq x \), \( P(T \leq t | X = x) = 0 \);

    When \( t > x \), \( P(T \leq t | X = x) = P(X + Y \leq t | X = x) = P(Y \leq t - x) = F_Y(t - x) = 1 - e^{-\lambda(t-x)} \).

    Therefore, the cumulative distribution function (CDF) is given by
    \[
    F_{T|X}(t | x) = 
    \begin{cases} 
    0, & t \leq x \\
    1 - e^{-\lambda(t-x)}, & t > x
    \end{cases}
    \]

    
    \item 
    The probability density function (PDF) is the derivative of the CDF with respect to \( t \):
    \[
    f_{T|X}(t | x) = \frac{\partial}{\partial t}(F_{T|X}(t | x)) = 
    \begin{cases} 
    0, & t \leq x \\
    \lambda e^{-\lambda(t-x)}, & t > x
    \end{cases}
    \]
    Therefore, \( f_{T|X}(t | x) \geq 0 \).

    The integral of the PDF over all \( t \) is 1, confirming that it is a valid PDF:
    \[
    \int_{-\infty}^{+\infty} f_{T|X}(t | x)dt = \int_{x}^{+\infty} \lambda e^{-\lambda(t-x)}dt = \left. -e^{-\lambda(t-x)} \right|_{t=x}^{+\infty} = 1.
    \]
    Therefore, it is a valid PDF.

    \item 
    Using Bayes' rule we have that
    \begin{align*}
    f_{X|T}(x|t) &= \frac{f(x, t)}{f_T(t)} = \frac{f_{T|X}(t|x)f_X(x)}{f_T(t)} \\
    &= \frac{\alpha e^{-\lambda(t-x)}e^{-\lambda x} \cdot \lambda x}{f_T(t)} \\
    &= \alpha \lambda^2 e^{-\lambda t} \cdot xt^{2x}
    \end{align*}
    for some \(\alpha > 0\). Observe that \(f_{X|T}(x|t)\) is a constant function respective to \(x\). In order to be a valid PDF, \(f_{X|T}(x|t)\) has to satisfy following
    \begin{equation*}
    1 = \int_{\mathbb{R}} f_{X|T}(x|t)dx = \int_{0}^{t} \alpha \lambda^2 e^{-\lambda t}dx = t\alpha \lambda^2 e^{-\lambda t}
    \end{equation*}
    So, for every \(t > 0\) there has to be
    \begin{equation*}
    \alpha = \frac{1}{t \lambda^2 e^{-\lambda t}}
    \end{equation*}
    and in this case it is a valid PDF.

    \item 
    Observe that in part (c) we have that in fact \(f_T(t) = \frac{1}{\alpha}\). So, we can easily obtain that
    \begin{equation*}
    f_T(t) = \lambda^2 t e^{-\lambda t}
    \end{equation*}

\end{enumerate}
\end{homeworkProblem}

\begin{homeworkProblem}[4]

    \begin{enumerate}[(a)]
    \item 
    Observe following for \( m \in (0,1) \)

    \[
    F_M(m) = P(M \leq m) = P(U_1 \leq m, \ldots, U_3 \leq m) = P(U_1 \leq m)\cdots P(U_3 \leq m) = m^3
    \]

    so the marginal PDF of \( M \) is

    \[
    f_M(m) = \frac{d}{dm}F_M(m) = 3m^2
    \]

    Now, let’s find joint CDF of \( L \) and \( M \). We have that

    \[
    P(L \geq l, M \leq m) = P(U_i \in [l, m], \forall i) = (m - l)^3
    \]

    for \( l \leq m \). Using the LOTP, we have that

    \[
    P(M \leq m) = P(M \leq m, L \leq l) + P(M \leq m, L > l)
    \]

    so use that to obtain

    \[
    F(m,l) = P(M \leq m, L \leq l) = m^3 - (m - l)^3
    \]

    for \( l \leq m \). Finally, the joint PDF is

    \[
    f(l,m) = \frac{\partial^2}{\partial m \partial l} F(m,l) = \frac{\partial^2}{\partial m \partial l} (m^3 - (m - l)^3) = 6(m - l)
    \]

    \item 
     Similarly as in (a) we can get that the CDF of \( L \) is \( F_L(l) = (1 - l)^3 \), so the PDF of \( L \) is \( f_L(l) = 3(1 - l)^2 \). So, using the definition of conditional PDF we get that

    \[
    f_{M|L}(m|l) = \frac{f(m,l)}{f_L(l)} = \frac{6(m - l)}{3(1 - l)^2} = \frac{2(m - l)}{(1 - l)^2}
    \]
    \end{enumerate}
\end{homeworkProblem}



\begin{homeworkProblem}[5]
\begin{enumerate}[(a)]
    
    \item
    \begin{align*}
    \bar{X} &= \frac{1}{n} \sum_{i=1}^{n} X_iY_i = \frac{1}{n} \sum_{i=1}^{n} y_i \\
    r &= \frac{1}{n} \sum_{i=1}^{n}[x_i - E(X)][y_i - E(Y)] \\
    &= E[(X - E(X))(Y - E(Y))] \\
    &= \text{Cov}(X,Y)
    \end{align*}

    \item
    \begin{align*}
    E((X - \bar{X})(Y - \bar{Y})) &= E(XY) - E(\bar{X}Y) - E(X\bar{Y}) + E(\bar{X}\bar{Y}) \\
    E(XY) &= \text{Cov}(X,Y) + E(X)E(Y) \\
    E(\bar{X}Y) &= \text{Cov}(\bar{X}, Y) + E(\bar{X})E(Y) = E(X)E(Y) \\
    E(X\bar{Y}) &= \text{Cov}(X\bar{Y}) + E(X)E(\bar{Y}) = E(X)E(Y) \\
    E(\bar{X}\bar{Y}) &= \text{Cov}(\bar{X}, \bar{Y}) + E(\bar{X})E(\bar{Y}) \\
    (X,Y) &\text{ is independent of } (\bar{X}, \bar{Y}) \\
    E(X) &= E(\bar{X}), \quad E(Y) = E(\bar{Y})
    \end{align*}

    \begin{align*}
    E((X - \bar{X})(Y - \bar{Y})) &= 2\text{Cov}(X, Y) \\
    S &= n^2(X_i - \bar{X})(Y_i - \bar{Y}) = 2n^2\text{Cov}(x,y) \\
    \text{Cov}(X,Y) &= \frac{S}{2n^2}
    \end{align*}

    \item
    \begin{enumerate}[1]
    \item After the horizontal and vertical coordinates are exchanged, the area will not change. If the area is unchanged, the covariance will not change.
    \item Enlarge the base vectors of the horizontal and vertical coordinates by $a_1$ and $a_2$ times respectively, and the area of the rectangle will become $a_1a_2$ times of the original one, so the covariance will also become $a_1a_2$ times of the original one.
    \item After moving the origin of the coordinate system to the left $a_1$ and down $a_2$, the area will not change. Therefore, the covariance will not change.
    \item The area of a rectangle with a length of $W_1$ and a height of $W_2 + W_3$ is equal to the sum of the areas of two rectangles with a length of $W_1$ and a width of $W_2$ and $W_3$ respectively, therefore $\text{Cov}(W_1, W_2 + W_3) = \text{Cov}(W_1, W_2) + \text{Cov}(W_1, W_3)$.
    \end{enumerate}

\end{enumerate}

\end{homeworkProblem}

\end{document}
