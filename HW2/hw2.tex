\documentclass{article}

\usepackage{fancyhdr}
\usepackage{extramarks}
\usepackage{amsmath}
\usepackage{amsthm}
\usepackage{amsfonts}
\usepackage{tikz}
\usepackage[plain]{algorithm}
\usepackage{algpseudocode}
\usepackage{enumerate}
\usepackage{tikz}
\usepackage{pythonhighlight}
\usetikzlibrary{automata,positioning}

%
% Basic Document Settings
%  

\topmargin=-0.45in
\evensidemargin=0in
\oddsidemargin=0in
\textwidth=6.5in
\textheight=9.0in
\headsep=0.25in

\linespread{1.1}

\pagestyle{fancy}
\lhead{\hmwkAuthorName}
\chead{\hmwkClass : \hmwkTitle}
\rhead{\firstxmark}
\lfoot{\lastxmark}
\cfoot{\thepage}

\renewcommand\headrulewidth{0.4pt}
\renewcommand\footrulewidth{0.4pt}

\setlength\parindent{0pt}

%
% Create Problem Sections
%

\newcommand{\enterProblemHeader}[1]{
    \nobreak\extramarks{}{Problem \arabic{#1} continued on next page\ldots}\nobreak{}
    \nobreak\extramarks{Problem \arabic{#1} (continued)}{Problem \arabic{#1} continued on next page\ldots}\nobreak{}
}

\newcommand{\exitProblemHeader}[1]{
    \nobreak\extramarks{Problem \arabic{#1} (continued)}{Problem \arabic{#1} continued on next page\ldots}\nobreak{}
    \stepcounter{#1}
    \nobreak\extramarks{Problem \arabic{#1}}{}\nobreak{}
}

\newcommand*\circled[1]{\tikz[baseline=(char.base)]{
		\node[shape=circle,draw,inner sep=2pt] (char) {#1};}}


\setcounter{secnumdepth}{0}
\newcounter{partCounter}
\newcounter{homeworkProblemCounter}
\setcounter{homeworkProblemCounter}{1}
\nobreak\extramarks{Problem \arabic{homeworkProblemCounter}}{}\nobreak{}

%
% Homework Problem Environment
%
% This environment takes an optional argument. When given, it will adjust the
% problem counter. This is useful for when the problems given for your
% assignment aren't sequential. See the last 3 problems of this template for an
% example.
%

\newenvironment{homeworkProblem}[1][-1]{
    \ifnum#1>0
        \setcounter{homeworkProblemCounter}{#1}
    \fi
    \section{Problem \arabic{homeworkProblemCounter}}
    \setcounter{partCounter}{1}
    \enterProblemHeader{homeworkProblemCounter}
}{
    \exitProblemHeader{homeworkProblemCounter}
}

%
% Homework Details
%   - Title
%   - Class
%   - Due date
%   - Name
%   - Student ID

\newcommand{\hmwkTitle}{Homework\ \#02}
\newcommand{\hmwkClass}{Probability \& Statistics for EECS}
\newcommand{\hmwkDueDate}{Mar 17, 2024}
\newcommand{\hmwkAuthorName}{Fei Pang}
\newcommand{\hmwkAuthorID}{2022533153}


%
% Title Page
%

\title{
    \vspace{2in}
    \textmd{\textbf{\hmwkClass:\\  \hmwkTitle}}\\
    \normalsize\vspace{0.1in}\small{Due\ on\ \hmwkDueDate\ at 23:59}\\
	\vspace{4in}
}

\author{
	Name: \textbf{\hmwkAuthorName} \\
	Student ID: \hmwkAuthorID}
\date{}

\renewcommand{\part}[1]{\textbf{\large Part \Alph{partCounter}}\stepcounter{partCounter}\\}

%
% Various Helper Commands
%

% Useful for algorithms
\newcommand{\alg}[1]{\textsc{\bfseries \footnotesize #1}}
% For derivatives
\newcommand{\deriv}[1]{\frac{\mathrm{d}}{\mathrm{d}x} (#1)}
% For partial derivatives
\newcommand{\pderiv}[2]{\frac{\partial}{\partial #1} (#2)}
% Integral dx
\newcommand{\dx}{\mathrm{d}x}
% Alias for the Solution section header
\newcommand{\solution}{\textbf{\large Solution}}
% Probability commands: Expectation, Variance, Covariance, Bias
\newcommand{\E}{\mathrm{E}}
\newcommand{\Var}{\mathrm{Var}}
\newcommand{\Cov}{\mathrm{Cov}}
\newcommand{\Bias}{\mathrm{Bias}}

\begin{document}

\maketitle

\pagebreak

\begin{homeworkProblem}[1]

\begin{enumerate}[(a)]
    % inline
    \item 
    B: Bob receives a 1.\\
    A: Alice sends a 1.\\
    So \[P(A|B) = \frac{P(B|A)P(A)}{P(B|A)P(A) + P(B|A^c)P(A^c)} = \frac{0.9 \cdot 0.5}{0.9 \cdot 0.5 + 0.05 \cdot 0.5} = \frac{18}{19}\] 
   
    \item 
    B: Bob receives a 110.\\
    A: Alice sends a 1.\\
    So \[P(A|B) = \frac{P(B|A)P(A)}{P(B|A)P(A) + P(B|A^c)P(A^c)} = \frac{0.9 \cdot 0.9 \cdot 0.05 \cdot 0.5}{0.9 \cdot 0.9 \cdot 0.05 \cdot 0.5 + 0.05 \cdot 0.05 \cdot 0.95 \cdot 0.5} = \frac{648}{667}\]

\end{enumerate}
\end{homeworkProblem}

\begin{homeworkProblem}[2]
\begin{enumerate}[(a)]

    \item
    T: He tests positive on n of the n tests.\\
    So \[P(D|T) = \frac{P(T|D)P(D)}{P(T|D)P(D) + P(T|D^c)P(D^c)} = \frac{a_0^np}{a_0^np + b_0^nq}\]

    \item
    T: He tests positive on n of the n tests.\\
    
    % So \ P(D|T) &= P(D|T,G)P(G) + P(D|T,G^c)P(G^c)\\ &= \frac{P(T,G|D)P(D)}{P(T,G|D)P(D) + P(T,G|D^c)P(D^c)} \cdot \frac{1}{2} + \frac{P(T,G^c|D)P(D)}{P(T,G^c|D)P(D) + P(T,G^c|D^c)P(D^c)} \cdot \frac{1}{2} \\
    % &= p \cdot \frac{1}{2} + \frac{a^np}{a^np + (1-p)b^n} \cdot \frac{1}{2} \\
    % &= \frac{1}{2}p + \frac{1}{2} \cdot \frac{a^np}{a^np + (1-p)b^n}\\
    % &= \frac{}{}
    \[P(D|T) = \frac{P(T|D)P(D)}{P(T|D)P(D)+P(T|D^c)P(D^c)}\]\\
    We need to calulate $P(T|D)$:\\
    \begin{align*}
        P(T|D) &= P(T|D,G)P(G) + P(T|D,G^c)P(G^c)\\
        &= \frac{1}{2} + \frac{1}{2}a_0^n
    \end{align*}
    Similarly, we have $P(T|D^C) = \frac{1}{2} + \frac{1}{2}b_0^n$.
    
    So 
    
    \[P(D|T) = \frac{p(\frac{1}{2} + \frac{1}{2}a_0^n)}{p(\frac{1}{2} + \frac{1}{2}a_0^n) + q(\frac{1}{2} + \frac{1}{2}b_0^n)}
        = \frac{p(1 + a_0^n)}{1 + a_0^np + b_0^nq}\]

    \end{enumerate}
\end{homeworkProblem}

\begin{homeworkProblem}[3]
    \begin{tiny}
    \begin{align*}
        &P(spam|W^C_1,\cdots,W_{22}^C,W_{23},W_{24}^C\cdots,W_{63}^C,W_{64},W_{65},\cdots,W_{66}^C,\cdots W_{100}^C)\\
        &=\frac{P(W^C_1,\cdots,W_{22}^C,W_{23},W_{24}^C\cdots,W_{63}^C,W_{64},W_{65},\cdots,W_{66}^C,\cdots W_{100}^C|spam)P(spam)}{P(W^C_1,\cdots,W_{22}^C,W_{23},W_{24}^C\cdots,W_{63}^C,W_{64},W_{65},W_{66}^C,\cdots W_{100}^C)}\\
        &=\frac{P(W^C_1,\cdots,W_{22}^C,W_{23},W_{24}^C\cdots,W_{63}^C,W_{64},W_{65},W_{66}^C,\cdots W_{100}^C|spam)P(spam)}{P(W^C_1,\cdots,W_{22}^C,W_{23},W_{24}^C\cdots,W_{63}^C,W_{64},W_{65},W_{66}^C,\cdots W_{100}^C|spam) P(spam)+P(W^C_1,\cdots,W_{22}^C,W_{23},W_{24}^C\cdots,W_{63}^C,W_{64},W_{65},\cdots W_{100}^C|not\ spam) P(not\ spam)}\\
        &=\frac{(1-p_1)\cdots (1-p_{22})p_{23}(1-p_{24})\cdots(1-p_{63})p_{64}p_{65}(1-p_{66})\cdots (1-p_{100})p}{(1-p_1)\cdots (1-p_{22})p_{23}(1-p_{24})\cdots(1-p_{63})p_{64}p_{65}(1-p_{66})\cdots (1-p_{100})p+(1-r_1)\cdots (1-r_{22})r_{23}(1-r_{24})\cdots(1-r_{63})r_{64}r_{65}(1-r_{66})\cdots (1-r_{100})(1-p)}
    \end{align*}
    \end{tiny}
    
\end{homeworkProblem}
\pagebreak

\begin{homeworkProblem}[4]
    Suppose we choose door 1. If we don't switch, the probability we win the car is $p1$.\\
    If we switch a door, as long as the car is not behind door 1, we'll win the car. So the probability is $1-p1$.\\
    If we choose door 2 or 3, that's the same.\\
    So we should choose door 3 and switch, so that the maximum probability that we win turn outs to be $p1+p2$.
\end{homeworkProblem}

\begin{homeworkProblem}[5]

\begin{enumerate}[(a)]
    \item 
    Let $C_j$ be the event that the car is hidden behind door $j$ and let $W$ be the event
    that we win using the switching strategy. Using the law of total probability, we
    can find the unconditional probability of winning:
    \[P(W) = P(W|C_1)P(C_1) + P(W|C_2)P(C_2) + P(W|C_3)P(C_3) = \frac{2}{3}\]

    \item 
    Let $D_i $be the event that Monty opens Door $i$.
    \begin{align*}
        P(C_3|D_2) &= \frac{P(D_2|C_3)P(C_3)}{P(D_2)}\\
        &= \frac{P(D_2|C_3)P(C_3)}{P(D_2|C_1)P(C_1) + P(D_2|C_2)P(C_2) + P(D_2|C_3)P(C_3)}\\
        &= \frac{1/3}{1/3 \cdot p + 1/3}\\
        &= \frac{1}{1 + p}
    \end{align*}

    \item 
    Replace $p$ by $1-p$.\\
    So \[P(C_2|D_3) = \frac{1}{2 - p}\]
\end{enumerate}
\end{homeworkProblem}


\begin{homeworkProblem}[6]
\begin{enumerate}[(a)]
    \item
    An: The nth trail is assigned treatment A.\\
    Bn: The nth trail is assigned treatment B.\\
    S: The nth trail is successful.\\

    \(p_n = p(S|A_n)p(A_n) + p(S|B_n)p(B_n) 
    = a_n \cdot a + (1 - a_n) \cdot b 
    = (a - b)a_n + b\)

    
    \(a_{n+1} = P(TreatmentA \ succeeds \ on \ the \ nth \ trial) + P(TreatmentB \ fails \ on \ the \ nth \ trial)\\
     = a_n \cdot a + (1-a_n)(1-b) \\
     = (a + b - 1)a_n + 1 - b\)

     \item 
    According to (a), we have:\\
    \begin{align*}
        p_{n + 1} &= (a - b)a_{n + 1} + b \\
        &= (a - b)[(a + b - 1)a_n + 1 - b] + b\\
        &= (a - b)(a + b - 1)a_n + ab + b^2 - b + a + b -2ab\\
        &= (a + b - 1)p_n + a + b - 2ab
    \end{align*}

    \item
    when $n \to \infty$, $ p_{n + 1} \approx p_n$ \\
    Assume \(p_{n + 1} \approx p_n = x\).\\
    So 
    \begin{align*}
    x &= (a + b - 1)x + a + b -2ab\\
    (2 - a - b)x &= a + b - 2ab\\
    x &= \frac{a + b - 2ab}{2 - a - b} 
    \end{align*}
    So \[\lim_{n \to \infty}p_n = \frac{a + b - 2ab}{2 - a - b}\]
\end{enumerate}

\end{homeworkProblem}
\end{document}
