\documentclass{article}

\usepackage{fancyhdr}
\usepackage{extramarks}
\usepackage{amsmath}
\usepackage{amsthm}
\usepackage{amsfonts}
\usepackage{tikz}
\usepackage[plain]{algorithm}
\usepackage{algpseudocode}
\usepackage{enumerate}
\usepackage{tikz}
\usepackage{pythonhighlight}
\usetikzlibrary{automata,positioning}

%
% Basic Document Settings
%  

\topmargin=-0.45in
\evensidemargin=0in
\oddsidemargin=0in
\textwidth=6.5in
\textheight=9.0in
\headsep=0.25in

\linespread{1.1}

\pagestyle{fancy}
\lhead{\hmwkAuthorName}
\chead{\hmwkClass : \hmwkTitle}
\rhead{\firstxmark}
\lfoot{\lastxmark}
\cfoot{\thepage}

\renewcommand\headrulewidth{0.4pt}
\renewcommand\footrulewidth{0.4pt}

\setlength\parindent{0pt}

%
% Create Problem Sections
%

\newcommand{\enterProblemHeader}[1]{
    \nobreak\extramarks{}{Problem \arabic{#1} continued on next page\ldots}\nobreak{}
    \nobreak\extramarks{Problem \arabic{#1} (continued)}{Problem \arabic{#1} continued on next page\ldots}\nobreak{}
}

\newcommand{\exitProblemHeader}[1]{
    \nobreak\extramarks{Problem \arabic{#1} (continued)}{Problem \arabic{#1} continued on next page\ldots}\nobreak{}
    \stepcounter{#1}
    \nobreak\extramarks{Problem \arabic{#1}}{}\nobreak{}
}

\newcommand*\circled[1]{\tikz[baseline=(char.base)]{
		\node[shape=circle,draw,inner sep=2pt] (char) {#1};}}


\setcounter{secnumdepth}{0}
\newcounter{partCounter}
\newcounter{homeworkProblemCounter}
\setcounter{homeworkProblemCounter}{1}
\nobreak\extramarks{Problem \arabic{homeworkProblemCounter}}{}\nobreak{}

%
% Homework Problem Environment
%
% This environment takes an optional argument. When given, it will adjust the
% problem counter. This is useful for when the problems given for your
% assignment aren't sequential. See the last 3 problems of this template for an
% example.
%

\newenvironment{homeworkProblem}[1][-1]{
    \ifnum#1>0
        \setcounter{homeworkProblemCounter}{#1}
    \fi
    \section{Problem \arabic{homeworkProblemCounter}}
    \setcounter{partCounter}{1}
    \enterProblemHeader{homeworkProblemCounter}
}{
    \exitProblemHeader{homeworkProblemCounter}
}

%
% Homework Details
%   - Title
%   - Class
%   - Due date
%   - Name
%   - Student ID

\newcommand{\hmwkTitle}{Homework\ \#05}
\newcommand{\hmwkClass}{Probability \& Statistics for EECS}
\newcommand{\hmwkDueDate}{Apr 7, 2024}
\newcommand{\hmwkAuthorName}{Fei Pang}
\newcommand{\hmwkAuthorID}{2022533153}


%
% Title Page
%

\title{
    \vspace{2in}
    \textmd{\textbf{\hmwkClass:\\  \hmwkTitle}}\\
    \normalsize\vspace{0.1in}\small{Due\ on\ \hmwkDueDate\ at 23:59}\\
	\vspace{4in}
}

\author{
	Name: \textbf{\hmwkAuthorName} \\
	Student ID: \hmwkAuthorID}
\date{}

\renewcommand{\part}[1]{\textbf{\large Part \Alph{partCounter}}\stepcounter{partCounter}\\}

%
% Various Helper Commands
%

% Useful for algorithms
\newcommand{\alg}[1]{\textsc{\bfseries \footnotesize #1}}
% For derivatives
\newcommand{\deriv}[1]{\frac{\mathrm{d}}{\mathrm{d}x} (#1)}
% For partial derivatives
\newcommand{\pderiv}[2]{\frac{\partial}{\partial #1} (#2)}
% Integral dx
\newcommand{\dx}{\mathrm{d}x}
% Alias for the Solution section header
\newcommand{\solution}{\textbf{\large Solution}}
% Probability commands: Expectation, Variance, Covariance, Bias
\newcommand{\E}{\mathrm{E}}
\newcommand{\Var}{\mathrm{Var}}
\newcommand{\Cov}{\mathrm{Cov}}
\newcommand{\Bias}{\mathrm{Bias}}

\begin{document}

\maketitle

\pagebreak

\begin{homeworkProblem}[1]
\begin{enumerate}[(a)]
    \item
    Define event $Z_1$: both nickles are Heads. So we have:
    \[Z_1 \sim Geom(p_1p_2)\]
    \[P(Z_1 = k) = (1-p_1p_2)^{k-1}p_1p_2\]
    \[E(Z_1) = \frac{1}{p_1p_2}\]

    \item 
    Define event $Z_2$: $\geq 1$ nickle is Head. So we have:
    \[Z_2 \sim Geom(1-(1-p_1)(1-p_2))\]
    \[E(Z_2) = \frac{1}{p_1+p_2-p_1p_2}\]

    \item 
    \[
    \mathbb{P}[X = Y] = \sum_{k=1}^{\infty} p^2 (q^2)^{k-1} = p^2 \sum_{k=0}^{\infty} (q^2)^k = \frac{p^2}{1 - q^2} = \frac{p}{2 - p}
    \]

    \[
    \mathbb{P}[X > Y] = \frac{1 - \frac{p}{2-p}}{2} = \cdots = \frac{1 - p}{2 - p}
    \]
\end{enumerate}
\end{homeworkProblem}
\begin{homeworkProblem}[2]
    \begin{enumerate}[(a)]
            
    \item
    Let $X$ be the random variable for the total amount of stops and let $I_k$ be the indicator variable such that $I_k = 1$ if someone stopped at the $k^{th}$ floor. Then

    \[
    X = I_2 + I_3 + \cdots + I_n,
    \]

    and it follows that:

    \[
    E[X] = E[I_2 + I_3 + \cdots + I_n] = E[I_2] + E[I_3] + \cdots + E[I_n].
    \]

    Then $E[I_j] = P(\text{at least someone hit the } j^{th} \text{ button}) = 1 - \left(\frac{n-2}{n-1}\right)^k$ 
    \[
        E[X] = (n - 1) \left(1 - \left(\frac{n-2}{n-1}\right)^k\right)
    \]

    \item 
    We use the same reasoning except we identify that

    \[
    E[I_j] = 1 - (1 - p_j)^k
    \]

    and it follows that
    \[
    E[X] = \sum_{j=2}^{n} 1 - (1 - p_j)^k = n - 1 - \sum_{j=2}^{n} (1 - p_j)^k.
    \]
    \end{enumerate}
\end{homeworkProblem}

\begin{homeworkProblem}[3]
    Using LOTUS, let $j = i-k$:
    \[
    E\binom{n}{k} = \sum_{i} f(i)P(X = i)
    \]
    \[
    = \sum_{i=k}^{\infty} \frac{i!}{k!(i - k)!} \cdot \frac{\lambda^i (e^{-\lambda})}{i!}
    \]
    \[
    = \lambda^k (e^{-\lambda}) \sum_{j=0}^{\infty} \frac{1}{k!} \cdot \frac{\lambda^j}{j!}
    \]
    \[
    = \lambda^k (e^{-\lambda}) \frac{1}{k!} e^{\lambda}
    \]
    \[
    = \frac{\lambda^k}{k!}
    \]
    
    
\end{homeworkProblem}

\begin{homeworkProblem}[4]
\begin{enumerate}[(a)]
    \item
    \[
    E(Xg(X)) = \sum_{k=0}^{\infty} kg(k) \frac{e^{-\lambda} \lambda^{k+1}}{k!} = \sum_{k=1}^{\infty} g(k) \frac{e^{-\lambda} \lambda^{k}}{(k - 1)!}
    \]

    \[
    \lambda E(g(X + 1)) = \lambda \sum_{k=0}^{\infty} g(k + 1) \frac{e^{-\lambda} \lambda^{k+1}}{k!} = \sum_{k=1}^{\infty} g(k) \frac{e^{-\lambda} \lambda^{k}}{(k - 1)!}
    \]

    \[
    E(Xg(X)) = \lambda E(g(X + 1))
    \]

    \item
    \[
    E(X^3) = E(X \cdot X^2) = \lambda E((X + 1)^2)
    \]
    \[
    \text{Var}(X + 1) = E((X + 1)^2) - E^2(X + 1)
    \]
    \[
    \lambda = E((X + 1)^2) - (\lambda + 1)^2
    \]
    \[
    E(X^3) = \lambda[\lambda + (\lambda + 1)^2]
    \]
    \[
    = \lambda^2 + \lambda(\lambda + 1)^2
    \]
    \[
    = \lambda^3 + 3\lambda^2 + \lambda
    \]

    \begin{align*}
    E(X^4) &= \lambda E[(X+1)^3]\\
    &= \lambda[E(X^3)+E(3X)+E(3X^2)+E(1)]\\
    &= \lambda(\lambda^3 + 3\lambda^2 + \lambda + 3(\lambda + \lambda^2)+3\lambda+1)\\
    &= \lambda^4 + 6\lambda^3 + 7\lambda^2 + \lambda
    \end{align*}

\end{enumerate}
\end{homeworkProblem}

\begin{homeworkProblem}[5]
    Let $p = q = \frac{1}{2}$ so that the expressions are close to those in class.\\
    Define $S_1$: the result of the first toss.\\
    Define $p_k = P(N = k)$\\
    So we have $p_0 = p_1 = p_2 = p_3 = 0$, $p_4 = pqpq$.\\
    For $k \geq 5$, we use the first-step method to analyze:
    \[p_k = P(N = k) = P(N = k, S_1 = H) + P(N = k, S_1 = T)\]
    We first deal with $P(N = K, S_1 = H)$:
    \begin{align*}
        P(N = k, S_1 = H) &= P(N = k, S_1 = H, S_2 = T) + P(N = k, S_1 = H, S_2 = H)\\
        &= P(N = k, S_1 = H, S_2 = T, S_3 = H) + P(N = k, S_1 = H, S_2 = T, S_3 = T) + p \cdot P(N = k - 1, S_2 = H)\\
        &= p^2q \cdot P(N = k - 3, S_3 = H) + pq^2 \cdot P(N = k - 3) + p \cdot P(N = k - 1, S_2 = H)
    \end{align*}

    To compute the form $P(N = k, S_1 = H)$:
    \[  P(N = k) = qP(N = k - 1) + P(N = k, S_1 = H) \]
    \[ P(N = k, S_1 = H) = P(N = k) -  qP(N = k - 1)\]
    So we have:
    \[ p^2q \cdot P(N = k - 3, S_3 = H) =  p^2q \cdot [P(N = k - 3) - qP(N = k - 4) ]\]
    \[ p \cdot P(N = k - 1, S_2 = H) = p \cdot [P(N = k - 1) - qP(N = k - 2)]\]
    Thus we obtain $P(N = K, S_1 = H)$:
    \[P(N = K, S_1 = H) = p^2q \cdot P(N = k - 3) - p^2q^2P(N = k - 4) + pq^2 \cdot P(N = k - 3) + p \cdot P(N = k - 1) - pq \cdot (N = k - 2)\]
    Also, we know $P(N = K, S_1 = T) = qP(N = k - 1)$.
    So we obtain:
    \[
        P(N = k) = p_{k-1} -pq \cdot p_{k-2} + pq \cdot p_{k-3} - p^2q^2 \cdot p_{k-4}
    \]
    According to Generating Function:
    \[g(t) = p^2q^2t^4+ \displaystyle \sum_{k=5}^{\infty} p_kt^k\]

    \[g(t) - p^2q^2t^4 =  (t - pqt^2 + pqt^3 - p^2q^2t^4)g(t)\]
    \[g(t) = \displaystyle \frac{p^2q^2t^4}{1 - t + pqt^2 - pqt^3 + p^2q^2t^4}\]

    Thus:
    \begin{align*}
    E(N) &= g'(t)\big|_{t = 1} = \displaystyle \frac{1}{p^2q^2} + \frac{1}{pq} = 20\\
    Var(N) &= g''(1) + g'(1) - [g'(1)]^2 = 276
\end{align*}
\end{homeworkProblem}

\end{document}
