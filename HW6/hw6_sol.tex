\documentclass{article}

\usepackage{fancyhdr}
\usepackage{extramarks}
\usepackage{amsmath}
\usepackage{amsthm}
\usepackage{amsfonts}
\usepackage{tikz}
\usepackage[plain]{algorithm}
\usepackage{algpseudocode}
\usepackage{enumerate}
\usepackage{tikz}
\usepackage{pythonhighlight}
\usetikzlibrary{automata,positioning}

%
% Basic Document Settings
%  

\topmargin=-0.45in
\evensidemargin=0in
\oddsidemargin=0in
\textwidth=6.5in
\textheight=9.0in
\headsep=0.25in

\linespread{1.1}

\pagestyle{fancy}
\lhead{\hmwkAuthorName}
\chead{\hmwkClass : \hmwkTitle}
\rhead{\firstxmark}
\lfoot{\lastxmark}
\cfoot{\thepage}

\renewcommand\headrulewidth{0.4pt}
\renewcommand\footrulewidth{0.4pt}

\setlength\parindent{0pt}

%
% Create Problem Sections
%

\newcommand{\enterProblemHeader}[1]{
    \nobreak\extramarks{}{Problem \arabic{#1} continued on next page\ldots}\nobreak{}
    \nobreak\extramarks{Problem \arabic{#1} (continued)}{Problem \arabic{#1} continued on next page\ldots}\nobreak{}
}

\newcommand{\exitProblemHeader}[1]{
    \nobreak\extramarks{Problem \arabic{#1} (continued)}{Problem \arabic{#1} continued on next page\ldots}\nobreak{}
    \stepcounter{#1}
    \nobreak\extramarks{Problem \arabic{#1}}{}\nobreak{}
}

\newcommand*\circled[1]{\tikz[baseline=(char.base)]{
		\node[shape=circle,draw,inner sep=2pt] (char) {#1};}}


\setcounter{secnumdepth}{0}
\newcounter{partCounter}
\newcounter{homeworkProblemCounter}
\setcounter{homeworkProblemCounter}{1}
\nobreak\extramarks{Problem \arabic{homeworkProblemCounter}}{}\nobreak{}

%
% Homework Problem Environment
%
% This environment takes an optional argument. When given, it will adjust the
% problem counter. This is useful for when the problems given for your
% assignment aren't sequential. See the last 3 problems of this template for an
% example.
%

\newenvironment{homeworkProblem}[1][-1]{
    \ifnum#1>0
        \setcounter{homeworkProblemCounter}{#1}
    \fi
    \section{Problem \arabic{homeworkProblemCounter}}
    \setcounter{partCounter}{1}
    \enterProblemHeader{homeworkProblemCounter}
}{
    \exitProblemHeader{homeworkProblemCounter}
}

%
% Homework Details
%   - Title
%   - Class
%   - Due date
%   - Name
%   - Student ID

\newcommand{\hmwkTitle}{Homework\ \#06}
\newcommand{\hmwkClass}{Probability \& Statistics for EECS}
\newcommand{\hmwkDueDate}{Apr 21, 2024}
\newcommand{\hmwkAuthorName}{Fei Pang}
\newcommand{\hmwkAuthorID}{2022533153}


%
% Title Page
%

\title{
    \vspace{2in}
    \textmd{\textbf{\hmwkClass:\\  \hmwkTitle}}\\
    \normalsize\vspace{0.1in}\small{Due\ on\ \hmwkDueDate\ at 23:59}\\
	\vspace{4in}
}

\author{
	Name: \textbf{\hmwkAuthorName} \\
	Student ID: \hmwkAuthorID}
\date{}

\renewcommand{\part}[1]{\textbf{\large Part \Alph{partCounter}}\stepcounter{partCounter}\\}

%
% Various Helper Commands
%

% Useful for algorithms
\newcommand{\alg}[1]{\textsc{\bfseries \footnotesize #1}}
% For derivatives
\newcommand{\deriv}[1]{\frac{\mathrm{d}}{\mathrm{d}x} (#1)}
% For partial derivatives
\newcommand{\pderiv}[2]{\frac{\partial}{\partial #1} (#2)}
% Integral dx
\newcommand{\dx}{\mathrm{d}x}
% Alias for the Solution section header
\newcommand{\solution}{\textbf{\large Solution}}
% Probability commands: Expectation, Variance, Covariance, Bias
\newcommand{\E}{\mathrm{E}}
\newcommand{\Var}{\mathrm{Var}}
\newcommand{\Cov}{\mathrm{Cov}}
\newcommand{\Bias}{\mathrm{Bias}}

\begin{document}

\maketitle

\pagebreak

\begin{homeworkProblem}[1]

    Let \( X \) be a continuous random variable with a PDF \( f \). In order to calculate the CDF of this random variable, we use the definition of CDF.

\[
F_X(x) = \int_{-\infty}^{x} f(\xi) \, d\xi = \int_{-\infty}^{x} \frac{1}{\pi(1 + \xi^2)} \, d\xi = \frac{1}{\pi} \int_{-\infty}^{x} \frac{1}{(1 + \xi^2)} \, d\xi
\]

\[
= \frac{1}{\pi} \left[ \tan^{-1}(\xi) \right]_{-\infty}^{x} = \frac{1}{\pi} \left[ \tan^{-1}(x) - \tan^{-1}\left(-\frac{\pi}{2}\right) \right]
\]

\[
= \frac{\tan^{-1}(x)}{\pi} + \frac{1}{2}
\]
\end{homeworkProblem}

\begin{homeworkProblem}[2]

For \( x \leq 1 \) we have that
\[
F(x) = P(X \leq x) = \int_{-\infty}^{x} f(u) du = \int_{-\infty}^{x} 0 du = 0
\]

Now, for \( x > 1 \) we have
\[
F(x) = P(X \leq x) = \int_{-\infty}^{x} f(u) du = \int_{1}^{x} \frac{a}{u^{a+1}} du = a \int_{1}^{x} u^{-a-1} du = a \left. \frac{u^{-a}}{-a} \right|_{1}^{x}
\]

\( \lim_{x \to \infty} F(x) = \lim_{x \to \infty} (1 - x^{-a}) = 1 \) and \( \lim_{x \to 0^+} F(x) = \lim_{x \to 0^+} (1 - x^{-a}) = 1 - F \) is certainly continuous 
on \( \mathbb{R} \) (eventual additional check is needed for \( c = 1 \) where we have \( \lim_{x \to 1^-} F(x) = 0 \) and \( \lim_{x \to 1^+} F(x) = 1 - 1 = 0 \)). So \( F \) is a valid CDF.

\end{homeworkProblem}

\begin{homeworkProblem}[3]
    \begin{enumerate}[(a)]
        \item 
        \[
        P(X \leq x) = \int_{-\infty}^{x} f(s) \, ds = \int_{0}^{x} f(s) \, ds = \int_{0}^{x} (12s^2 - 12s^3) \, ds = 4x^3 - 3x^4
        \]
        
        \item 
        Using part (a), we have following
        \[
        P(0 < X < 1/2) = F(1/2) - F(0) = \frac{5}{16}
        \]

        \item 
        \[
        E(X) = \int_{\mathbb{R}} x f(x) \, dx = \int_{0}^{1} x (12x^2 - 12x^3) \, dx = \int_{0}^{1} 12x^3 - 12x^4 \, dx = \frac{3}{5}
        \]
        and
        \[
        E(X^2) = \int_{\mathbb{R}} x^2 f(x) \, dx = \int_{0}^{1} x^2 (12x^2 - 12x^3) \, dx = \int_{0}^{1} 12x^4 - 12x^5 \, dx = \frac{2}{5}
        \]
        Finally, we have
        \[
        \text{Var}(X) = E(X^2) - (EX)^2 = \frac{2}{5} - \left( \frac{3}{5} \right)^2 = \frac{2}{5} - \frac{9}{25} = \frac{1}{25}
        \]
    \end{enumerate}
\end{homeworkProblem}

\begin{homeworkProblem}[4]
    \begin{enumerate}[(a)]
        \item 
        Suppose that there was \(G\) failures before first success. Each failure has taken \(\Delta t\) time to happen. At the time when \(G\)th failure happen, 
        we have already pasted \((G - 1)\Delta t\) time. Then, there has to pass \(\Delta t\) time and then we have successful trial. Finally, we have that
        \[
        T = (G - 1)\Delta t + \Delta t = G\Delta t
        \]
        
        \item
        Use part (a) to obtain following
        \[
        P(T > t) = P(G\Delta t > t) = P\left( G > \frac{t}{\Delta t} \right) = (1 - \lambda\Delta t)^\frac{t}{\Delta t}
        \]
        
        \[
        P(T \leq t) = 1 - P(T > t) = 1 - (1 - \lambda\Delta t)^\frac{t}{\Delta t}
        \]

        \item 
        Let's fix \( t \geq 0 \). Use the definition of exponential function to obtain that
        \[
        \lim_{\Delta t \to 0} P(T \leq t) = \lim_{\Delta t \to 0} \left[ 1 - \left( 1 - \lambda \Delta t \right)^\frac{t}{\Delta t} \right]
        \]
        \[
        = \lim_{\Delta t \to 0} \left[ 1 - \left( (1 + \frac{-\lambda}{\frac{1}{\Delta t}} )^\frac{1}{\Delta t}\right)^{t}\right] = 1 - e^{-\lambda t}
        \]
        Observe that on limit we have got that CDF of \(T\) is exactly PDF of exponential distribution with parameter \(\lambda\).
    \end{enumerate}

\end{homeworkProblem}

\begin{homeworkProblem}[5]

    Observe that it can be written as following
    \[
    \max(Z - c, 0) = (Z - c)\mathbf{x}_{\{Z-c>0\}} = (Z - c)\mathbf{x}_{\{Z>c\}}
    \]
    So using LOTUS, we have following
    \[
    E(\max(Z - c, 0)) = \int_{\mathbb{R}} (z - c)\mathbf{x}_{\{z>c\}}f(z) \, dz = \int_{c}^{\infty} (z - c)f(z) \, dz
    \]
    \[
    = \int_{c}^{\infty} zf(z) \, dz - \int_{c}^{\infty} cf(z) \, dz = I_1 - I_2
    \]
    Let's calculate first integral. We have following
    \[
    I_1 = \int_{c}^{\infty} \frac{z}{\sqrt{2\pi}} e^{-\frac{z^2}{2}} \, dz
    \]
    \[
    = \int_{\frac{c^2}{2}}^{\infty} \frac{1}{\sqrt{2\pi}} e^{-u} \, du = \frac{1}{\sqrt{2\pi}} e^{-\frac{c^2}{2}} = \varphi(c)
    \]
    For the second integral, we have
    \[
    I_2 = cP(Z > c) = c(1 - P(Z \leq c)) = c(1 - \Phi(z))
    \]
    Finally, we have obtained that
    \[
    E(\max(Z - c, 0)) = \varphi(c) - c(1 - \Phi(z))
    \]
\end{homeworkProblem}

\end{document}
