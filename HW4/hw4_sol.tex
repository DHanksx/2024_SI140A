\documentclass{article}

\usepackage{fancyhdr}
\usepackage{extramarks}
\usepackage{amsmath}
\usepackage{amsthm}
\usepackage{amsfonts}
\usepackage{tikz}
\usepackage[plain]{algorithm}
\usepackage{algpseudocode}
\usepackage{enumerate}
\usepackage{tikz}
\usepackage{pythonhighlight}
\usetikzlibrary{automata,positioning}

%
% Basic Document Settings
%  

\topmargin=-0.45in
\evensidemargin=0in
\oddsidemargin=0in
\textwidth=6.5in
\textheight=9.0in
\headsep=0.25in

\linespread{1.1}

\pagestyle{fancy}
\lhead{\hmwkAuthorName}
\chead{\hmwkClass : \hmwkTitle}
\rhead{\firstxmark}
\lfoot{\lastxmark}
\cfoot{\thepage}

\renewcommand\headrulewidth{0.4pt}
\renewcommand\footrulewidth{0.4pt}

\setlength\parindent{0pt}

%
% Create Problem Sections
%

\newcommand{\enterProblemHeader}[1]{
    \nobreak\extramarks{}{Problem \arabic{#1} continued on next page\ldots}\nobreak{}
    \nobreak\extramarks{Problem \arabic{#1} (continued)}{Problem \arabic{#1} continued on next page\ldots}\nobreak{}
}

\newcommand{\exitProblemHeader}[1]{
    \nobreak\extramarks{Problem \arabic{#1} (continued)}{Problem \arabic{#1} continued on next page\ldots}\nobreak{}
    \stepcounter{#1}
    \nobreak\extramarks{Problem \arabic{#1}}{}\nobreak{}
}

\newcommand*\circled[1]{\tikz[baseline=(char.base)]{
		\node[shape=circle,draw,inner sep=2pt] (char) {#1};}}


\setcounter{secnumdepth}{0}
\newcounter{partCounter}
\newcounter{homeworkProblemCounter}
\setcounter{homeworkProblemCounter}{1}
\nobreak\extramarks{Problem \arabic{homeworkProblemCounter}}{}\nobreak{}

%
% Homework Problem Environment
%
% This environment takes an optional argument. When given, it will adjust the
% problem counter. This is useful for when the problems given for your
% assignment aren't sequential. See the last 3 problems of this template for an
% example.
%

\newenvironment{homeworkProblem}[1][-1]{
    \ifnum#1>0
        \setcounter{homeworkProblemCounter}{#1}
    \fi
    \section{Problem \arabic{homeworkProblemCounter}}
    \setcounter{partCounter}{1}
    \enterProblemHeader{homeworkProblemCounter}
}{
    \exitProblemHeader{homeworkProblemCounter}
}

%
% Homework Details
%   - Title
%   - Class
%   - Due date
%   - Name
%   - Student ID

\newcommand{\hmwkTitle}{Homework\ \#04}
\newcommand{\hmwkClass}{Probability \& Statistics for EECS}
\newcommand{\hmwkDueDate}{Mar 31, 2024}
\newcommand{\hmwkAuthorName}{Fei Pang}
\newcommand{\hmwkAuthorID}{2022533153}


%
% Title Page
%

\title{
    \vspace{2in}
    \textmd{\textbf{\hmwkClass:\\  \hmwkTitle}}\\
    \normalsize\vspace{0.1in}\small{Due\ on\ \hmwkDueDate\ at 23:59}\\
	\vspace{4in}
}

\author{
	Name: \textbf{\hmwkAuthorName} \\
	Student ID: \hmwkAuthorID}
\date{}

\renewcommand{\part}[1]{\textbf{\large Part \Alph{partCounter}}\stepcounter{partCounter}\\}

%
% Various Helper Commands
%

% Useful for algorithms
\newcommand{\alg}[1]{\textsc{\bfseries \footnotesize #1}}
% For derivatives
\newcommand{\deriv}[1]{\frac{\mathrm{d}}{\mathrm{d}x} (#1)}
% For partial derivatives
\newcommand{\pderiv}[2]{\frac{\partial}{\partial #1} (#2)}
% Integral dx
\newcommand{\dx}{\mathrm{d}x}
% Alias for the Solution section header
\newcommand{\solution}{\textbf{\large Solution}}
% Probability commands: Expectation, Variance, Covariance, Bias
\newcommand{\E}{\mathrm{E}}
\newcommand{\Var}{\mathrm{Var}}
\newcommand{\Cov}{\mathrm{Cov}}
\newcommand{\Bias}{\mathrm{Bias}}

\begin{document}

\maketitle

\pagebreak

\begin{homeworkProblem}[1]


    \[
    E(X) = \sum_{k=1}^{\infty} k P(X = k) = c \sum_{k=1}^{\infty}p^k
    = c \left( \sum_{k=0}^{\infty} p^k - p^0 \right)
    = c \left( \frac{p}{(1-p)^2} \right)
    \]
    
    We use formula \( \text{Var}(X) = E(X^2) - (E(X))^2 \). 
    To calculate \( E(X^2) \):
    
    \[
    E(X(X - 1)) = \sum_{k=1}^{\infty} k(k - 1) P(X = k) 
    = \sum_{k=1}^{\infty} k(k - 1) c \frac{p^k}{k} 
    = c \sum_{k=1}^{\infty} (k - 1) p^k
    \]
    
    \[
    = c p^2 \sum_{k=0}^{\infty} k p^{k-1}
    = c p^2 \frac{d}{dp} \left( \sum_{k=0}^{\infty} p^k \right)
    = c p^2 \frac{d}{dp} \left( \frac{1}{1 - p} \right)
    = \frac{c p^2}{(1 - p)^2}
    \]
    
   To obtain \( E(X^2) \):
    
    \[
    E(X^2) = E(X(X - 1)) + E(X) 
    = \frac{cp^2}{(1 - p)^2} + \frac{cp}{1 - p}
    = \frac{cp^2 + cp(1 - p)}{(1 - p)^2}
    \]
    
   Compute the variance:
    
    \[
    \text{Var}(X) = E(X^2) - (E(X))^2 
    = \frac{cp^2 + cp(1 - p)}{(1 - p)^2} - \left( \frac{cp}{1 - p} \right)^2
    \]
    
\end{homeworkProblem}

\begin{homeworkProblem}[2]
    Let \(N = w + b\), \(p = w/N\), \(q = 1 - p\).
    \begin{enumerate}[(a)]
    
    \item
    So \(X \sim HGeom(w, b, n)\).

    \[P(X = k) =  \frac{\binom{w}{k} \binom{b}{n - k}}{\binom{w + b}{n}}\]
    
    \[E\binom{X}{2} = \binom{n}{2}\frac{w}{w + b}\frac{w - 1}{w + b - 1}\]


    \item 
    By (a),
    \begin{equation*}
    EX^2 - EX = E(X(X - 1)) = n(n - 1)p\frac{w - 1}{N - 1},
    \end{equation*}
    so
    \begin{equation*}
        \begin{aligned}
        Var(X) &= E(X^2) - (EX)^2 \\
        &= n(n - 1)p\frac{w - 1}{N - 1} + np - n^2p^2 \\
        &= np\left(\frac{(n - 1)(w - 1)}{N - 1} + 1 - np\right) \\
        &= np\left(\frac{nw - w - n + N}{N - 1} - \frac{nw}{N}\right) \\
        &= np\left(\frac{Nnw - Nw - Nn + N^2 - Nnw + nw}{N(N - 1)}\right) \\
        &= np\left(\frac{(N - n)(N - w)}{N(N - 1)}\right) \\
        &= \frac{N - n}{N - 1}npq.
        \end{aligned}
    \end{equation*}
    \end{enumerate}
\end{homeworkProblem}


\begin{homeworkProblem}[3]

    $N$: \# of toys needed to obtain all types of toys.\\
    $N = N_1 + N_2 + \cdots + N_n$\\
    $N_1 = 1$\\
    $N_2 \sim Fs(\displaystyle \frac{n-1}{n}), N_3 \sim Fs(\frac{n-2}{n}) \cdots$
    So \(E(N_j) = \displaystyle \frac{n}{n-(j-1)}\).\\
    \(E(N) = E(N_1) + E(N_2) + \cdots + E(N_n) = n \displaystyle \sum_{j = 1}^{n} \frac{1}{j}\).\\
    since $Y \sim Fs(p)$:\\
    \[E(Y) = \displaystyle \frac{1}{p}, Var(Y) = \frac{1-p}{p^2}\].
    \[Var(N_j) = \displaystyle \frac{1-\frac{n-(j-1)}{n}}{[\frac{n-(j-1)}{n}]^2} = \frac{n(j-1)}{(n-(j-1))^2}\].\\
    \[Var(N) = Var(N_1) + Var(N_2) + \cdots + Var(N_n) = n \cdot \displaystyle \sum_{j = 1}^{n}\frac{j-1}{(n-(j-1))^2}\]
    
    


\end{homeworkProblem}

\begin{homeworkProblem}[4]
    
    \(I_j: \text{Type } j \text{ toy occurs in } m \text{ collected toys.} \quad E(I_j) = 1 - (1 - p_j)^m. \\
    N = I_1 + I_2 + \cdots + I_n\)
    \[E(N) = \sum_{i=1}^{n} E(1 - (1 - p_i)^m) = n - \sum_{i=1}^{n} (1 - p_i)^m\]
    \[E(N^2) = E\left((I_1 + \cdots + I_n)^2\right)\\
    = \sum_{j=1}^{n} E(I_j) + 2 \sum_{i<j} E(I_i I_j) \\
    = n - \sum_{i=1}^{n} (1 - p_i)^m + 2 \sum_{i<j} (1 - (1 - p_i)^m)(1 - (1 - p_j)^m)\]
    So 
    \[\text{Var}(N) = E(N^2) - E^2(N) = n - \sum_{i=1}^{n} (1 - p_i)^m + 2 \sum_{i<j} (1 - (1 - p_i)^m)(1 - (1 - p_j)^m) - \left(n - \sum_{i=1}^{n} (1 - p_i)^m\right)^2 \]        
\end{homeworkProblem}


\begin{homeworkProblem}[5]

    \begin{enumerate}[(a)]
        \item
        \[P(X \geq 24) = 1 - P(X \leq 23) = 0.493 < \displaystyle \frac{1}{2}\]
        Thus, $m \geq 24$ cannot be the median. Similarly:\\
        \[P(X \leq 22) = 1 - P(X \geq 23) < \displaystyle \frac{1}{2}\] 
        So $m = 23$ is the unique median.
    
        \item 
        Suppose that $X=k(1\leq k\leq 366)$, it means that no birthday match among $k-1$ people.
        $Therefore I_1, I_2, \cdots , I_k=1$, $I_{k+1},I_{k+2},\cdots,I_{366}=0$.
        $Therefore I_1+ I_2+ \cdots+I_{366}=k $
        $Therefore X=I_1+ I_2+ \cdots+I_{366}$
        \[E(X) = \displaystyle \sum_{j=1}^{366}E(I_j) = \sum_{j=1}^{366}P(I_j=1) = \sum_{j=1}^{366}p_j\]
        
        \item
        For the set of keys {[1, 4, 5, 10, 16, 17, 21]}, draw binary search tree of height 3, 4, 5, and 6.
        Use python to calculate:
        \begin{python}
    
            EX = 0
            for j in range(1, 367):
                pj = 1
                if j < 3:
                    pj = 1
                if j > 2:
                    for i in range(3, j + 1):
                        pj = pj * (1 - (i - 2) / 365)
                    EX = EX + pj
            print(EX)
        \end{python}
        \[E(X) \approx 24.617\]
    
        \item 
    
        \[I_i^2 = I_i, \ I_iI_j = 1\]
        \[
        X^2 = I_1 + \ldots + I_{366} + 2 \sum_{j=3}^{366} \sum_{i=1}^{j-2} I_i I_j
        \]
    
        \[
        = I_1 + \ldots + I_{366} + 2 \sum_{j=3}^{366} (j - 2)I_j
        \]
    
        \[
        = \sum_{j=1}^{366} (2j - 1)I_j
        \]
    
        \[
        E(X^2) = \sum_{j=1}^{366} (2j - 1)E(I_j)
        \]
    
        \[
        = \sum_{j=1}^{366} (2j - 1)P(I_j)
        \]
    
        \[
        P(I_j) = E(I_j) = P(I_j = 1) = p_j
        \]
    
        \[
        E(X^2) = \sum_{j=1}^{366} (2j - 1)p_j
        \]
    
        again, we use python:
        \begin{python}
            EX2 = 0
            for j in range(1, 367):
                pj = 1
                if j < 3:
                    pj = 1
                if j > 2:
                    for i in range(3, j+1):
                        pj = pj * (1 - (i - 2)/365)
                EX2 = EX2 + (2*j - 1) * pj
        
        \end{python}
        So
        \[E(X^2) \approx 754.617\]
        \[Var(X) = E(X^2) - (EX)^2 \approx 148.640\]
        \end{enumerate}
    
\end{homeworkProblem}

\end{document}
