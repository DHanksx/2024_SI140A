\documentclass{article}

\usepackage{fancyhdr}
\usepackage{extramarks}
\usepackage{amsmath}
\usepackage{amsthm}
\usepackage{amsfonts}
\usepackage{tikz}
\usepackage[plain]{algorithm}
\usepackage{algpseudocode}
\usepackage{enumerate}
\usepackage{tikz}
\usepackage{pythonhighlight}
\usetikzlibrary{automata,positioning}

%
% Basic Document Settings
%  

\topmargin=-0.45in
\evensidemargin=0in
\oddsidemargin=0in
\textwidth=6.5in
\textheight=9.0in
\headsep=0.25in

\linespread{1.1}

\pagestyle{fancy}
\lhead{\hmwkAuthorName}
\chead{\hmwkClass : \hmwkTitle}
\rhead{\firstxmark}
\lfoot{\lastxmark}
\cfoot{\thepage}

\renewcommand\headrulewidth{0.4pt}
\renewcommand\footrulewidth{0.4pt}

\setlength\parindent{0pt}

%
% Create Problem Sections
%

\newcommand{\enterProblemHeader}[1]{
    \nobreak\extramarks{}{Problem \arabic{#1} continued on next page\ldots}\nobreak{}
    \nobreak\extramarks{Problem \arabic{#1} (continued)}{Problem \arabic{#1} continued on next page\ldots}\nobreak{}
}

\newcommand{\exitProblemHeader}[1]{
    \nobreak\extramarks{Problem \arabic{#1} (continued)}{Problem \arabic{#1} continued on next page\ldots}\nobreak{}
    \stepcounter{#1}
    \nobreak\extramarks{Problem \arabic{#1}}{}\nobreak{}
}

\newcommand*\circled[1]{\tikz[baseline=(char.base)]{
		\node[shape=circle,draw,inner sep=2pt] (char) {#1};}}


\setcounter{secnumdepth}{0}
\newcounter{partCounter}
\newcounter{homeworkProblemCounter}
\setcounter{homeworkProblemCounter}{1}
\nobreak\extramarks{Problem \arabic{homeworkProblemCounter}}{}\nobreak{}

%
% Homework Problem Environment
%
% This environment takes an optional argument. When given, it will adjust the
% problem counter. This is useful for when the problems given for your
% assignment aren't sequential. See the last 3 problems of this template for an
% example.
%

\newenvironment{homeworkProblem}[1][-1]{
    \ifnum#1>0
        \setcounter{homeworkProblemCounter}{#1}
    \fi
    \section{Problem \arabic{homeworkProblemCounter}}
    \setcounter{partCounter}{1}
    \enterProblemHeader{homeworkProblemCounter}
}{
    \exitProblemHeader{homeworkProblemCounter}
}

%
% Homework Details
%   - Title
%   - Class
%   - Due date
%   - Name
%   - Student ID

\newcommand{\hmwkTitle}{Homework\ \#03}
\newcommand{\hmwkClass}{Probability \& Statistics for EECS}
\newcommand{\hmwkDueDate}{Mar 24, 2024}
\newcommand{\hmwkAuthorName}{Fei Pang}
\newcommand{\hmwkAuthorID}{2022533153}


%
% Title Page
%

\title{
    \vspace{2in}
    \textmd{\textbf{\hmwkClass:\\  \hmwkTitle}}\\
    \normalsize\vspace{0.1in}\small{Due\ on\ \hmwkDueDate\ at 23:59}\\
	\vspace{4in}
}

\author{
	Name: \textbf{\hmwkAuthorName} \\
	Student ID: \hmwkAuthorID}
\date{}

\renewcommand{\part}[1]{\textbf{\large Part \Alph{partCounter}}\stepcounter{partCounter}\\}

%
% Various Helper Commands
%

% Useful for algorithms
\newcommand{\alg}[1]{\textsc{\bfseries \footnotesize #1}}
% For derivatives
\newcommand{\deriv}[1]{\frac{\mathrm{d}}{\mathrm{d}x} (#1)}
% For partial derivatives
\newcommand{\pderiv}[2]{\frac{\partial}{\partial #1} (#2)}
% Integral dx
\newcommand{\dx}{\mathrm{d}x}
% Alias for the Solution section header
\newcommand{\solution}{\textbf{\large Solution}}
% Probability commands: Expectation, Variance, Covariance, Bias
\newcommand{\E}{\mathrm{E}}
\newcommand{\Var}{\mathrm{Var}}
\newcommand{\Cov}{\mathrm{Cov}}
\newcommand{\Bias}{\mathrm{Bias}}

\begin{document}

\maketitle

\pagebreak

\begin{homeworkProblem}[1]

B: The villagers believe the boy.\\
A: The $i$th time that the boy said "A wolf comes".\\
Suppose \(P(B) = p_1, P(A_i|B) = p_2, P(A_i|B^c) = p_3, p_3 > p_2\)
    \begin{align*}
        P(B|A_1) &= \frac{P(A_1|B)P(B)}{P(A_1|B)P(B) + P(A_1|B^c)P(B^c)}\\
        &= \frac{p_2p_1}{p_2p_1 + p_3(1 - p_1)}\\
        &= \frac{p_1}{p_1 + \frac{p_3}{p_2}(1 - p_1)}
    \end{align*}
Similar, we have:\\
\begin{align*}
    P(B|A_1, A_2) &= \frac{p_1}{p_1 + (\frac{p_3}{p_2})^2(1 - p_1)}\\
    P(B|A_1, A_2, A_3) &= \frac{p_1}{p_1 + (\frac{p_3}{p_2})^3(1 - p_1)}
\end{align*}
Let \(\frac{p_3}{p_2} = x (x > 1)\):\\
\[P(B|A_1, A_2, \cdots, A_k) = \frac{p_1}{p_1 + (\frac{p_3}{p_2})^k(1 - p_1)}\]
Thus, when \(k \uparrow, P(B)\downarrow\).
\end{homeworkProblem}

\begin{homeworkProblem}[2]
    \begin{enumerate}[(a)]
        \item
        In previous throw we have to have a sum $n-6$ and then get number 6 or we have to have a sum $n-5$ and get number 5 and so on.\\
        Each number is equally likely, so we can write\[p_n = 1/6 \cdot p_{n - 1} + 1/6 \cdot p_{n - 2} + \cdots + 1/6 \cdot p_{n - 6}\] 
        \item
        \[
            \begin{aligned}
            p_1 &= \frac{1}{6} \\
            p_2 &= \frac{1}{6} \left(1 + \frac{1}{6}\right) \\
            p_3 &= \frac{1}{6} \left(1 + \frac{1}{6}\right)^2 = \frac{1}{6} \left(1 + \frac{1}{6}\right)^2 \\
            p_4 &= \frac{1}{6} \left(1 + \frac{1}{6}\right)^3 \\
            &\vdots \\
            p_7 &= \frac{1}{6}(p_1 + p_2 + p_3 + p_4 + p_5 + p_6 - 1) \\
                &= \frac{1}{6} \left(\left(1 + \frac{1}{6}\right)^6 - 1\right)
            \end{aligned}
        \]

        \item
        In every throw, there in average falls \(\frac{1+2+3+4+5+6}{6} = \frac{7}{2}\). So we can expect that for each number except that in seven throws that
        number will fall twice, therefore the required probability is \(\frac{2}{7}\).

    \end{enumerate}
\end{homeworkProblem}

\begin{homeworkProblem}[3]
    \begin{enumerate}[(a)]
    \item
    $S_i$: The trail succeeds in the $i$th.
    \[P(A_2) = P(S_1^c \cap S_2^c) + P(S_1 \cap S_2) = q_1q_2 + (1-q_1)(1-q_2) = 2q_1q_2 - (q_1 + q_2) + \frac{1}{2} + \frac{1}{2} = 2b_1b_2 + \frac{1}{2}\].
    
    \item
    It can be seen easily that:
    \begin{align*}
    P(A_{n+1}) &= P(A_n)(\frac{1}{2} + b_{n+1}) + (1-P_{A_n})(\frac{1}{2} - b_{n+1})\\
    &= (2P_{A_n} - 1)b_{n+1} + \frac{1}{2}
    \end{align*}
    
    Since \(P(A_n) = \frac{1}{2} + 2^{n-1}b_1b_2 \cdots b_n\):
    \[P(A_{n+1}) = \frac{1}{2} + 2^nb_1b_2 \cdots b_nb_{n+1}\]
    Thus the induction shows that the equation is right.
    
    \item
    \begin{enumerate}
        \item When $P_i = \frac{1}{2}$, since the probability of success and failure are at the same. $P$(the number of successful trials is even) $= P$(the number of successful trials is odd) $=\frac{1}{2}$
        
         $P_i = \frac{1}{2} \implies q_i = \frac{1}{2}, b_i = 0 \Rightarrow P(A_n) = \frac{1}{2} \Rightarrow$ \textit{correct}
    
        \item When $P_i = 1$, $A_n$ directly depends on $n$'s odd or even $\Rightarrow P(A_n) = \begin{cases} 
            0 & n \text{ is odd} \\
            1 & n \text{ is even} 
        \end{cases}$

         $P_i = 1 \implies b_i = -\frac{1}{2} \implies P(A_n) = \begin{cases} 
            0 & n \text{ is odd} \\
            1 & n \text{ is even} 
        \end{cases} \Rightarrow$ \textit{correct}
    
        \item When $P_i = 0$ number of success trial is always 0
        
         $P_i = 0 \implies b_i = \frac{1}{2} \implies P(A_n) = 1 \Rightarrow$ \textit{correct}
    \end{enumerate}
\end{enumerate}

\end{homeworkProblem}



\begin{homeworkProblem}[4]
    \begin{enumerate}
        \item[(a)]solution:
        \[
        P = \binom{5}{2}p^2(1-p)^3 + \binom{5}{4}p^4(1-p)
        \]
        \[
        = 10 \times 0.01 \times 0.729 + 5 \times 0.0001 \times 0.9
        \]
        \[
        = 0.07335
        \]
        
        \item[(b)]solution:
        \[
        P = \sum_{\substack{k \text{ is even}, k \geq 2}}^{n} \binom{n}{k} p^k (1 - p)^{n-k}
        \] 
        
        \item[(c)]solution:
        
        \(a = \sum\limits_{\substack{k \text{ is even}, k \geq 0}}^{n} \binom{n}{k} p^k (1 - p)^{n-k}\)\\
        \(b = \sum\limits_{\substack{k \text{ is odd}, k \geq 1}}^{n} \binom{n}{k} p^k (1 - p)^{n-k}\)\\
        Using Binomial theorem, observe that we have
        \[
        a + b = \sum_{k=0}^{n} \binom{n}{k} p^k (1 - p)^{n-k} = 1
        \]
        \[
        a - b = \sum_{k=0}^{n} \binom{n}{k} (-p)^k (1 - p)^{n-k} = (1 - 2p)^n
        \]

        Solve this system of two equations with variables to obtain that:
        \[
        a = \frac{1 + (1 - 2p)^n}{2}, \quad b = \frac{1 - (1 - 2p)^n}{2}
        \]

        Finally, we have
        \[
        \sum_{\substack{k \geq 2, \\ k \text{ even}}} \binom{n}{k} p^k (1 - p)^{n-k} = \sum_{\substack{k \geq 0, \\ k \text{ even}}} \binom{n}{k} p^k (1 - p)^{n-k} - (1 - p)^n
        \]
        \[
        = \frac{1 + (1 - 2p)^n}{2} - (1 - p)^n
        \]

    \end{enumerate}
\end{homeworkProblem}

\pagebreak
\begin{homeworkProblem}[5]

\begin{enumerate}[(a)]

    \item
    \begin{align*}
        P(X \oplus Y = 1) &= p \times \frac{1}{2} + (1 - p) \times \frac{1}{2} = \frac{1}{2} \\
        P(X \oplus Y = 0) &= \frac{1}{2} \\
        X \oplus Y &\sim \mathit{Bern}(\frac{1}{2})\\
    \end{align*} 

    \item
    \begin{align*}
        P(X \oplus Y = 0|X = 0) &= \left(1-p\right) \times \frac{1}{2} \times \frac{1}{1-p} = \frac{1}{2} \\
        P(X \oplus Y = 0|X = 1) &= p \times \frac{1}{2} \times \frac{1}{p} = \frac{1}{2} \\
        P(X \oplus Y = 1|X = 0) &= \left(1-p\right) \times \frac{1}{2} \times \frac{1}{1-p} = \frac{1}{2} \\
        P(X \oplus Y = 1|X = 1) &= p \times \frac{1}{2} \times \frac{1}{p} = \frac{1}{2} \\
        \end{align*}
        whether \( p \) is \( \frac{1}{2} \) or not \\
        \( X \oplus Y \) is independent of \( X \) \\
        \( P(X \oplus Y = 0|Y = 0) = \frac{1}{2} \times p \neq P(X \oplus Y = 0) = \frac{1}{2} \) \\
        if \( p \neq \frac{1}{2}, X \oplus Y \) is dependent of \( Y \) \\
        if \( p = \frac{1}{2}, X \oplus Y \) is independent of \( Y \)

    \item
        we know that \( X_j \) that equals 1 when there are odd \( X_j = 1 \)
        
        \[P(Y_j = 1) = P(\text{odd } X_i = 1 \text{ in } X_n)\]
        in Problem 3 we know that\( P(odd  X_j = 1  in X_n) = \frac{1}{2} \text{ when } P(x_j = 1) = \frac{1}{2} \) \\
        So\\
        \[P(Y_J = 1)= \frac{1}{2}, \text{ and } Y_J \sim \text{Bern}\left(\frac{1}{2}\right)\]
      
        then proof that pairwise independent
        \begin{align*}
        P(Y_{J_a} = 1|Y_{J_b} = 1) &= \frac{P(Y_{J_a} = 1 \text{ and } Y_{J_b} = 1)}{P(Y_{J_b} = 1)} \\
        C &= J_a \cap J_b, C \text{ can be } \emptyset, D = \{x_i : x_i \in J_a, x_i \not\in J_b\} 
        \end{align*}
        when  C  isn't $\emptyset$\\
        \begin{align*}
        P(Y_{J_a} = 1 \text{ and } Y_{J_b} = 1) &= P(Y_{J_b} = 1)P(Y_{J_c} = 1)P(Y_{J_d} = 0) + P(Y_{J_b} = 1)P(Y_{J_c} = 0)P(Y_{J_d} = 1) \\
        &= \frac{1}{2} \times \frac{1}{2} \times \frac{1}{2} \times \frac{1}{2} = \frac{1}{4} \\
        \end{align*}
        when  C  is $\emptyset$\\
        \begin{align*}
        P(Y_{J_a} = 1 \text{ and } Y_{J_b} = 1) &= P(Y_{J_b} = 1)P(Y_{J_d} = 1) \\
        &= \frac{1}{2} \times \frac{1}{2} = \frac{1}{4} \\
        \end{align*}
        \begin{align*}
        P(Y_{J_a} = 0|Y_{J_b} = 0) &= \frac{1}{2} \\
        P(Y_{J_a} = 1|Y_{J_b} = 0) &= \frac{1}{2} \\
        P(Y_{J_a} = 0|Y_{J_b} = 1) &= \frac{1}{2} \\
        P(Y_{J_a} = 1|Y_{J_b} = 1) &= \frac{1}{2} \\
        \end{align*}
        Similarly,
        \begin{align*}
        P(Y_{J_a} = 0|Y_{J_b} = 0) &= \frac{1}{2} \\
        P(Y_{J_a} = 1|Y_{J_b} = 0) &= \frac{1}{2} \\
        P(Y_{J_a} = 0|Y_{J_b} = 1) &= \frac{1}{2} \\
        P(Y_{J_a} = 1|Y_{J_b} = 1) &= \frac{1}{2} \\
        \end{align*}
        So pairwise independent. Then prove not independent.
        \begin{align*}
        \text{If } Y_{J_1} = 1, Y_{J_2} = 1, \text{ then } Y_{J_3} = 0, \text{ where } J_1 = \{X_1\}, J_2 = \{X_2\}, J_3 = \{X_1, X_2\} \\
        P(Y_{J_3} = 0|Y_{J_1} = 1, Y_{J_2} = 1) &\neq P(Y_{J_3} = 0)
        \end{align*}
        So not independent.

    \end{enumerate}
\end{homeworkProblem}

\end{document}
