\documentclass{article}

\usepackage{fancyhdr}
\usepackage{extramarks}
\usepackage{amsmath}
\usepackage{amsthm}
\usepackage{amsfonts}
\usepackage{tikz}
\usepackage[plain]{algorithm}
\usepackage{algpseudocode}
\usepackage{enumerate}
\usepackage{tikz}
\usepackage{pythonhighlight}
\usetikzlibrary{automata,positioning}

%
% Basic Document Settings
%  

\topmargin=-0.45in
\evensidemargin=0in
\oddsidemargin=0in
\textwidth=6.5in
\textheight=9.0in
\headsep=0.25in

\linespread{1.1}

\pagestyle{fancy}
\lhead{\hmwkAuthorName}
\chead{\hmwkClass : \hmwkTitle}
\rhead{\firstxmark}
\lfoot{\lastxmark}
\cfoot{\thepage}

\renewcommand\headrulewidth{0.4pt}
\renewcommand\footrulewidth{0.4pt}

\setlength\parindent{0pt}

%
% Create Problem Sections
%

\newcommand{\enterProblemHeader}[1]{
    \nobreak\extramarks{}{Problem \arabic{#1} continued on next page\ldots}\nobreak{}
    \nobreak\extramarks{Problem \arabic{#1} (continued)}{Problem \arabic{#1} continued on next page\ldots}\nobreak{}
}

\newcommand{\exitProblemHeader}[1]{
    \nobreak\extramarks{Problem \arabic{#1} (continued)}{Problem \arabic{#1} continued on next page\ldots}\nobreak{}
    \stepcounter{#1}
    \nobreak\extramarks{Problem \arabic{#1}}{}\nobreak{}
}

\newcommand*\circled[1]{\tikz[baseline=(char.base)]{
		\node[shape=circle,draw,inner sep=2pt] (char) {#1};}}


\setcounter{secnumdepth}{0}
\newcounter{partCounter}
\newcounter{homeworkProblemCounter}
\setcounter{homeworkProblemCounter}{1}
\nobreak\extramarks{Problem \arabic{homeworkProblemCounter}}{}\nobreak{}

%
% Homework Problem Environment
%
% This environment takes an optional argument. When given, it will adjust the
% problem counter. This is useful for when the problems given for your
% assignment aren't sequential. See the last 3 problems of this template for an
% example.
%

\newenvironment{homeworkProblem}[1][-1]{
    \ifnum#1>0
        \setcounter{homeworkProblemCounter}{#1}
    \fi
    \section{Problem \arabic{homeworkProblemCounter}}
    \setcounter{partCounter}{1}
    \enterProblemHeader{homeworkProblemCounter}
}{
    \exitProblemHeader{homeworkProblemCounter}
}

%
% Homework Details
%   - Title
%   - Class
%   - Due date
%   - Name
%   - Student ID

\newcommand{\hmwkTitle}{Homework\ \#08}
\newcommand{\hmwkClass}{Probability \& Statistics for EECS}
\newcommand{\hmwkDueDate}{May 5, 2024}
\newcommand{\hmwkAuthorName}{Fei Pang}
\newcommand{\hmwkAuthorID}{2022533153}


%
% Title Page
%

\title{
    \vspace{2in}
    \textmd{\textbf{\hmwkClass:\\  \hmwkTitle}}\\
    \normalsize\vspace{0.1in}\small{Due\ on\ \hmwkDueDate\ at 23:59}\\
	\vspace{4in}
}

\author{
	Name: \textbf{\hmwkAuthorName} \\
	Student ID: \hmwkAuthorID}
\date{}

\renewcommand{\part}[1]{\textbf{\large Part \Alph{partCounter}}\stepcounter{partCounter}\\}

%
% Various Helper Commands
%

% Useful for algorithms
\newcommand{\alg}[1]{\textsc{\bfseries \footnotesize #1}}
% For derivatives
\newcommand{\deriv}[1]{\frac{\mathrm{d}}{\mathrm{d}x} (#1)}
% For partial derivatives
\newcommand{\pderiv}[2]{\frac{\partial}{\partial #1} (#2)}
% Integral dx
\newcommand{\dx}{\mathrm{d}x}
% Alias for the Solution section header
\newcommand{\solution}{\textbf{\large Solution}}
% Probability commands: Expectation, Variance, Covariance, Bias
\newcommand{\E}{\mathrm{E}}
\newcommand{\Var}{\mathrm{Var}}
\newcommand{\Cov}{\mathrm{Cov}}
\newcommand{\Bias}{\mathrm{Bias}}

\begin{document}

\maketitle

\pagebreak

\begin{homeworkProblem}[1]
\begin{enumerate}[(a)]
    \item
   
    \[
    1 = \int_{-\infty}^{+\infty} \int_{-\infty}^{+\infty} f_{X,Y}(x, y) \, dy \, dx
    = \int_0^1 \int_0^x cx^2 y \, dy \, dx
    = \int_0^1 \frac{c x^4}{2} \, dx
    = \frac{c}{10}
    \]
    So \(c = 10\).

    \item 
    \[
    P \left( Y \leq \frac{X}{4} \mid Y \leq \frac{X}{2} \right) = \frac{P(Y \leq \frac{X}{4}, Y \leq \frac{X}{2})}{P(Y \leq \frac{X}{2})}
    = \frac{P(Y \leq \frac{X}{4})}{P(Y \leq \frac{X}{2})}
    = \frac{\int_0^1 \int_0^{\frac{x}{4}} 10x^2 y \, dy \, dx}{\int_0^1 \int_0^{\frac{x}{2}} 10x^2 y \, dy \, dx}
    = \frac{1}{4}
    \]


    
\end{enumerate}
\end{homeworkProblem}


\begin{homeworkProblem}[2]
\begin{enumerate}[(a)]
    \item 
    The marginal distributions of \(X\) is 
    \[
    P_X(X) = \sum_{y=0}^{\infty} P_{X,Y}(X, Y).
    \]
    \(X = 0\):
    \[
    P(X = 0) = P(X = 0, Y = 0) + P(X = 0, Y = 1) = \frac{1}{3}.
    \]
    \(X \neq 0\):
    \[
    P(X = x) = P(X = x, Y = x - 1) + P(X = x, Y = x) + P(X = x, Y = x + 1) = \frac{1}{6 \cdot 2^{x-2}} 
    \]
    The marginal distribution of \(X\) is 
    \[
    P_X(X) = 
    \begin{cases} 
    \frac{1}{3}, & x = 0 \\
    \frac{1}{6 \cdot 2^{x-2}}, & x > 0 \\
    0, & \text{otherwise}.
    \end{cases}
    \]
    According to the symmetric property:
    \[
    P_Y(Y) = 
    \begin{cases} 
    \frac{1}{3}, & y = 0 \\
    \frac{1}{6 \cdot 2^{x-2}} , & y > 0 \\
    0, & \text{otherwise}.
    \end{cases}
    \]

    \item 
    
    \[
    P_{X,Y}(0,0) = \frac{1}{6},  P(X = 0)P(Y = 0) = \frac{1}{9}
    \]
    
    So \(X\) and \(Y\) are not independent.

    \item 
    Since symmetric, we have \(P(X = Y)\) = \(P(X = Y - 1)\) = \(P(X = Y + 1)\) and \(P(X = Y)\) + \(P(X = Y - 1)\) + \(P(X = Y + 1) = 1\). So 
    \[
    P(X = Y) = \frac{1}{3}.
    \]

\end{enumerate}
\end{homeworkProblem}


\begin{homeworkProblem}[3]
\begin{enumerate}[(a)]
    \item 
    Yes, this random vector is multivariate Normal because for any \(a, b, c \in \mathbb{R}\) we have that
    \[
    aX + bY + c(X + Y) = (a + c)X + (b + c)Y
    \]
    and any linear combination of independent normally distributed variables is again Normal.

    \item 
    Let's consider random variable \(Z = X + Y + SX + SY = (1+S)X + (1+S)Y\). Observe that event \(Z = 0\) is in fact event \(S = -1\). Hence, we have that
    \[
    P(Z = 0) = P(S = -1) = \frac{1}{2}
    \]
    On the other hand, none of the normally distributed random variable has probability \(\frac{1}{2}\) in zero. Hence, \(Z\) is not normally distributed. So, we have found one linear combination that is not Normal. Thus, given random vector is not multivariate Normal.

    \item 
    Observe that random vector \((X, Y)\) is identically distributed as \((-X, -Y)\). So, take any \(C \in \mathbb{R}\). We have following
    
    \begin{align*}
    P((SX, SY) \in C) &= P((X,Y) \in C, S = 1) + P((-X, -Y) \in C, S = -1) \\
    &= P((X,Y) \in C \mid S = 1)P(S = 1) + P((-X, -Y) \in C \mid S = -1)P(S = -1) \\
    &= \frac{1}{2}P((X,Y) \in C) + \frac{1}{2}P((-X, -Y) \in C) = P((X,Y) \in C)
    \end{align*}
    
    So, we obtained that \((SX, SY)\) is equally distributed as \((X, Y)\). Also, we know that \((X, Y)\) is Bivariate normal. Hence, so is \((SX, SY)\).

\end{enumerate}
\end{homeworkProblem}

\begin{homeworkProblem}[4]
\begin{enumerate}[(a)]
    \item 
    For \(a, b \in \mathbb{R}\), we have
    \[
    aX + bY = (a\Sigma_X + b\Sigma_Y \rho)Z_1 + b\sqrt{1 - \rho^2} \Sigma_Y Z_2 + a\mu_X + b\mu_Y.
    \]
    Since the linear combination of two Normal distribution follows Normal distribution, \(X\) and \(Y\) are bivariate normal.

    \item 
    Since \(Z_1, Z_2 \sim N(0,1)\), we have \(Z_1 + \sqrt{1-\rho^2} Z_2 \sim N(0,1)\). So \(X \sim N(\mu_X, \Sigma_X)\), \(Y \sim N(\mu_Y, \Sigma_Y)\). Thus, we have
    \[
    \text{Cov}(X,Y) = \text{Cov}(\Sigma_X Z_1 + \mu_X, \Sigma_Y(\rho Z_1 + \sqrt{1-\rho^2} Z_2) + \mu_Y)
    = \Sigma_X \Sigma_Y \text{Cov}(Z_1, \rho Z_1 + \sqrt{1-\rho^2} Z_2)
    = \Sigma_X \Sigma_Y \rho.
    \]
    Then correlation coefficient between \(X\) and \(Y\) is
    \[
    \text{Corr}(X,Y) = \frac{\text{Cov}(X,Y)}{\sqrt{\text{Var}(X) \text{Var}(Y)}} = \frac{\Sigma_X \Sigma_Y \rho}{\Sigma_X \Sigma_Y} = \rho.
    \]

    \item 
    Since \(Z_1\) and \(Z_2\) are i.i.d., we have
    \[
    f_{Z_1,Z_2}(z_1, z_2) = f_{Z_1}(z_1) f_{Z_2}(z_2) = \frac{1}{2\pi} e^{-\frac{z_1^2 + z_2^2}{2}}.
    \]
    Since \(X = \Sigma_X Z_1 + \mu_X\), \(Y = \Sigma_Y(\rho Z_1 + \sqrt{1-\rho^2} Z_2) + \mu_Y\), we have
    \[
    Z_1 = \frac{X - \mu_X}{\Sigma_X}, \quad Z_2 = \frac{Y - \mu_Y - \rho \Sigma_Y Z_1}{\sqrt{1-\rho^2} \Sigma_Y} = \frac{Y - \mu_Y}{\sqrt{1-\rho^2} \Sigma_Y} - \frac{\rho (X - \mu_X)}{\sqrt{1-\rho^2} \Sigma_X}.
    \]

    \[
    f_{X,Y}(x, y) = \left|\frac{\partial(Z_1, Z_2)}{\partial(X, Y)}\right| f_{Z_1,Z_2}(z_1, z_2)
    \]
    \[
    =\frac{1}{2\pi \Sigma_X \Sigma_Y \sqrt{1 - \rho^2}} \exp \left( -\frac{1}{2(1-\rho^2)} \left[ \left(\frac{x-\mu_X}{\Sigma_X}\right)^2 + \frac{2\rho(x-\mu_X)(y-\mu_Y)}{\sigma_X\sigma_Y} - \left(\frac{y-\mu_Y}{\Sigma_Y}\right)^2 \right] \right).
    \]
\end{enumerate}

\end{homeworkProblem}


\begin{homeworkProblem}[5]
\begin{enumerate}[(a)]
 
    \item 
    \(x = zy\), so
    \begin{align*}
        f_Z(z) &= \int_{-\infty}^{\infty} f_{Z|Y}(z|y) f_Y(y) \, dy \\
               &= \int_{-\infty}^{\infty} f_X(zy) f_Y(y)y \, dy \\
               &= \frac{1}{4\pi} \int_{-\infty}^{\infty} e^{-\frac{1}{2} y^2 (z^2 + 1)} \, dy^2 \\
               &= \frac{1}{\pi (1 + z^2)}.
    \end{align*}
        
    \item 
    Given that \(X\) and \(Y\) are independent and uniformly distributed over \([0, 1]\), the joint PDF of \(W = XY\) and \(Z = \frac{X}{Y}\) is derived as follows:

    \[
    f_{W,Z}(w, z) = f_{X,Y}(x, y) \left| \frac{1}{\frac{\partial(w, z)}{\partial(x, y)}} \right|
    \]

    Where the Jacobian determinant \(\left| \frac{\partial(w, z)}{\partial(x, y)} \right|\) is calculated by:
    \[
    \left| \frac{\partial(w, z)}{\partial(x, y)} \right| = \left| \begin{array}{cc}
    \frac{\partial w}{\partial x} & \frac{\partial w}{\partial y} \\
    \frac{\partial z}{\partial x} & \frac{\partial z}{\partial y} 
    \end{array} \right|
    = \left| \begin{array}{cc}
    y & x \\
    \frac{1}{y} & -\frac{x}{y^2}
    \end{array} \right|
    = -\frac{2x}{y}
    \]

    \[
    f_{W,Z}(w, z) = \frac{y}{2x}
    \]
    
    \item 
    \[
    f_{X,R}(x, r) = f_{X|R}(x|r) f_R(r)
    \]

    We know from the area that the CDF of \( R \) is
    \[
    F_R(r) = r^2
    \]
    So
    \[
    f_R(r) = \frac{d}{dr} F_R(r) = 2r
    \]

    We know that \( X = R\cos\Theta \), where \( \Theta \sim \text{Unif}(0, 2\pi) \). Thus \( f_\Theta(\theta) = \frac{1}{2\pi} \).

    Thus,
    \[
    f_{X|R}(x|r) = f_{\Theta|R}(\theta|r) \left|\frac{dx}{d\theta}\right| = \frac{1}{2\pi |r\sin\theta|} = \frac{1}{2\pi \sqrt{r^2 - x^2}}
    \]

    Thus,
    \[
    f_{X,R}(x, r) = \frac{1}{2\pi \sqrt{r^2 - x^2}} \cdot 2r = \frac{r}{\pi \sqrt{r^2 - x^2}} \quad \text{for} \quad |x| < r
    \]

    Thus
    \[
    f_{X,R}(x, r) = 
    \begin{cases} 
    \frac{r}{\pi \sqrt{r^2 - x^2}} & \text{if } |x| < r \\
    0 & \text{otherwise}
    \end{cases}
    \]

    \item 
    \[
    F_R(r) = \frac{4\pi r^3}{3}
    \]
    Thus,
    \[
    f_R(r) = \frac{d}{dr} F_R(r) = 3r^2
    \]
    Let \( x = r \sin \theta \cos \alpha \), \( y = r \sin \theta \sin \alpha \), \( z = r \cos \theta \), so
    \[
    f_{Z|R}(z|r) = \frac{1}{2\pi \sqrt{r^2 - z^2}}
    \]

    Thus,
    \[
    f_{Z,R}(z, r) = \frac{1}{2\pi \sqrt{r^2 - z^2}} \cdot 3r^2 = \frac{3r^2}{2\pi \sqrt{r^2 - z^2}}
    \]

    So,
    \[
    f_{Z,R}(z, r) = 
    \begin{cases} 
    \frac{3r^2}{2\pi \sqrt{r^2 - z^2}}, & \text{if } |z| < r \\
    0, & \text{otherwise}
    \end{cases}
    \]



\end{enumerate}

\end{homeworkProblem}

\end{document}
